% Options for packages loaded elsewhere
\PassOptionsToPackage{unicode}{hyperref}
\PassOptionsToPackage{hyphens}{url}
%
\documentclass[
  letterpaper,
  paper=6in:9in,
  pagesize=pdftex,
  headinclude=on,
  footinclude=on,
  12pt]{scrbook}

\usepackage{amsmath,amssymb}
\usepackage{iftex}
\ifPDFTeX
  \usepackage[T1]{fontenc}
  \usepackage[utf8]{inputenc}
  \usepackage{textcomp} % provide euro and other symbols
\else % if luatex or xetex
  \usepackage{unicode-math}
  \defaultfontfeatures{Scale=MatchLowercase}
  \defaultfontfeatures[\rmfamily]{Ligatures=TeX,Scale=1}
\fi
\usepackage{lmodern}
\ifPDFTeX\else  
    % xetex/luatex font selection
\fi
% Use upquote if available, for straight quotes in verbatim environments
\IfFileExists{upquote.sty}{\usepackage{upquote}}{}
\IfFileExists{microtype.sty}{% use microtype if available
  \usepackage[]{microtype}
  \UseMicrotypeSet[protrusion]{basicmath} % disable protrusion for tt fonts
}{}
\makeatletter
\@ifundefined{KOMAClassName}{% if non-KOMA class
  \IfFileExists{parskip.sty}{%
    \usepackage{parskip}
  }{% else
    \setlength{\parindent}{0pt}
    \setlength{\parskip}{6pt plus 2pt minus 1pt}}
}{% if KOMA class
  \KOMAoptions{parskip=half}}
\makeatother
\usepackage{xcolor}
\setlength{\emergencystretch}{3em} % prevent overfull lines
\setcounter{secnumdepth}{5}
% Make \paragraph and \subparagraph free-standing
\makeatletter
\ifx\paragraph\undefined\else
  \let\oldparagraph\paragraph
  \renewcommand{\paragraph}{
    \@ifstar
      \xxxParagraphStar
      \xxxParagraphNoStar
  }
  \newcommand{\xxxParagraphStar}[1]{\oldparagraph*{#1}\mbox{}}
  \newcommand{\xxxParagraphNoStar}[1]{\oldparagraph{#1}\mbox{}}
\fi
\ifx\subparagraph\undefined\else
  \let\oldsubparagraph\subparagraph
  \renewcommand{\subparagraph}{
    \@ifstar
      \xxxSubParagraphStar
      \xxxSubParagraphNoStar
  }
  \newcommand{\xxxSubParagraphStar}[1]{\oldsubparagraph*{#1}\mbox{}}
  \newcommand{\xxxSubParagraphNoStar}[1]{\oldsubparagraph{#1}\mbox{}}
\fi
\makeatother


\providecommand{\tightlist}{%
  \setlength{\itemsep}{0pt}\setlength{\parskip}{0pt}}\usepackage{longtable,booktabs,array}
\usepackage{calc} % for calculating minipage widths
% Correct order of tables after \paragraph or \subparagraph
\usepackage{etoolbox}
\makeatletter
\patchcmd\longtable{\par}{\if@noskipsec\mbox{}\fi\par}{}{}
\makeatother
% Allow footnotes in longtable head/foot
\IfFileExists{footnotehyper.sty}{\usepackage{footnotehyper}}{\usepackage{footnote}}
\makesavenoteenv{longtable}
\usepackage{graphicx}
\makeatletter
\newsavebox\pandoc@box
\newcommand*\pandocbounded[1]{% scales image to fit in text height/width
  \sbox\pandoc@box{#1}%
  \Gscale@div\@tempa{\textheight}{\dimexpr\ht\pandoc@box+\dp\pandoc@box\relax}%
  \Gscale@div\@tempb{\linewidth}{\wd\pandoc@box}%
  \ifdim\@tempb\p@<\@tempa\p@\let\@tempa\@tempb\fi% select the smaller of both
  \ifdim\@tempa\p@<\p@\scalebox{\@tempa}{\usebox\pandoc@box}%
  \else\usebox{\pandoc@box}%
  \fi%
}
% Set default figure placement to htbp
\def\fps@figure{htbp}
\makeatother

\usepackage{fvextra}
\DefineVerbatimEnvironment{Highlighting}{Verbatim}{breaklines,commandchars=\\\{\}}
\areaset[0.50in]{4.5in}{8in}
\makeatletter
\@ifpackageloaded{bookmark}{}{\usepackage{bookmark}}
\makeatother
\makeatletter
\@ifpackageloaded{caption}{}{\usepackage{caption}}
\AtBeginDocument{%
\ifdefined\contentsname
  \renewcommand*\contentsname{Inhaltsverzeichnis}
\else
  \newcommand\contentsname{Inhaltsverzeichnis}
\fi
\ifdefined\listfigurename
  \renewcommand*\listfigurename{Abbildungsverzeichnis}
\else
  \newcommand\listfigurename{Abbildungsverzeichnis}
\fi
\ifdefined\listtablename
  \renewcommand*\listtablename{Tabellenverzeichnis}
\else
  \newcommand\listtablename{Tabellenverzeichnis}
\fi
\ifdefined\figurename
  \renewcommand*\figurename{Abbildung}
\else
  \newcommand\figurename{Abbildung}
\fi
\ifdefined\tablename
  \renewcommand*\tablename{Tabelle}
\else
  \newcommand\tablename{Tabelle}
\fi
}
\@ifpackageloaded{float}{}{\usepackage{float}}
\floatstyle{ruled}
\@ifundefined{c@chapter}{\newfloat{codelisting}{h}{lop}}{\newfloat{codelisting}{h}{lop}[chapter]}
\floatname{codelisting}{Listing}
\newcommand*\listoflistings{\listof{codelisting}{Listingverzeichnis}}
\makeatother
\makeatletter
\makeatother
\makeatletter
\@ifpackageloaded{caption}{}{\usepackage{caption}}
\@ifpackageloaded{subcaption}{}{\usepackage{subcaption}}
\makeatother

\ifLuaTeX
\usepackage[bidi=basic]{babel}
\else
\usepackage[bidi=default]{babel}
\fi
\babelprovide[main,import]{ngerman}
% get rid of language-specific shorthands (see #6817):
\let\LanguageShortHands\languageshorthands
\def\languageshorthands#1{}
\ifLuaTeX
  \usepackage[german]{selnolig} % disable illegal ligatures
\fi
\usepackage{bookmark}

\IfFileExists{xurl.sty}{\usepackage{xurl}}{} % add URL line breaks if available
\urlstyle{same} % disable monospaced font for URLs
\hypersetup{
  pdftitle={Book Title},
  pdfauthor={KING},
  pdflang={de},
  hidelinks,
  pdfcreator={LaTeX via pandoc}}


\title{Book Title}
\author{KING}
\date{2024-12-06}

\begin{document}
\frontmatter
\maketitle

\RecustomVerbatimEnvironment{verbatim}{Verbatim}{
   showspaces = false,
   showtabs = false,
   breaksymbolleft={},
   breaklines
}

\renewcommand*\contentsname{Inhaltsverzeichnis}
{
\setcounter{tocdepth}{2}
\tableofcontents
}

\mainmatter
\bookmarksetup{startatroot}

\chapter*{KINGsNotese}\label{kingsnotese}
\addcontentsline{toc}{chapter}{KINGsNotese}

\markboth{KINGsNotese}{KINGsNotese}

\bookmarksetup{startatroot}

\chapter*{Introduction To this
template}\label{introduction-to-this-template}
\addcontentsline{toc}{chapter}{Introduction To this template}

\markboth{Introduction To this template}{Introduction To this template}

This document serves as a guide for a project designed to demonstrate
the process of generating PDF documents from Markdown using Quarto and
LaTeX. Additionally, it incorporates the use of GitHub Actions to
automate the generation process and GitHub Releases for storing the
final output. This repository has been structured as a GitHub Template,
allowing it to be easily used as a starting point for any book project
or documentation endeavor.

\part{Linux}

This document serves as a guide for a project designed to demonstrate
the process of generating PDF documents from Markdown using Quarto and
LaTeX. Additionally, it incorporates the use of GitHub Actions to
automate the generation process and GitHub Releases for storing the
final output. This repository has been structured as a GitHub Template,
allowing it to be easily used as a starting point for any book project
or documentation endeavor.

\chapter*{Commands}\label{commands}
\addcontentsline{toc}{chapter}{Commands}

\markboth{Commands}{Commands}

This document serves as a guide for a project designed to demonstrate
the process of generating PDF documents from Markdown using Quarto and
LaTeX. Additionally, it incorporates the use of GitHub Actions to
automate the generation process and GitHub Releases for storing the
final output. This repository has been structured as a GitHub Template,
allowing it to be easily used as a starting point for any book project
or documentation endeavor.

\part{DevOps}

This document serves as a guide for a project designed to demonstrate
the process of generating PDF documents from Markdown using Quarto and
LaTeX. Additionally, it incorporates the use of GitHub Actions to
automate the generation process and GitHub Releases for storing the
final output. This repository has been structured as a GitHub Template,
allowing it to be easily used as a starting point for any book project
or documentation endeavor.

\chapter{Containers}\label{containers}

This document serves as a guide for a project designed to demonstrate
the process of generating PDF documents from Markdown using Quarto and
LaTeX. Additionally, it incorporates the use of GitHub Actions to
automate the generation process and GitHub Releases for storing the
final output. This repository has been structured as a GitHub Template,
allowing it to be easily used as a starting point for any book project
or documentation endeavor.


\backmatter


\end{document}
