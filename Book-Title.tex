% Options for packages loaded elsewhere
\PassOptionsToPackage{unicode}{hyperref}
\PassOptionsToPackage{hyphens}{url}
%
\documentclass[
  letterpaper,
  paper=6in:9in,
  pagesize=pdftex,
  headinclude=on,
  footinclude=on,
  12pt]{scrbook}

\usepackage{amsmath,amssymb}
\usepackage{iftex}
\ifPDFTeX
  \usepackage[T1]{fontenc}
  \usepackage[utf8]{inputenc}
  \usepackage{textcomp} % provide euro and other symbols
\else % if luatex or xetex
  \usepackage{unicode-math}
  \defaultfontfeatures{Scale=MatchLowercase}
  \defaultfontfeatures[\rmfamily]{Ligatures=TeX,Scale=1}
\fi
\usepackage{lmodern}
\ifPDFTeX\else  
    % xetex/luatex font selection
\fi
% Use upquote if available, for straight quotes in verbatim environments
\IfFileExists{upquote.sty}{\usepackage{upquote}}{}
\IfFileExists{microtype.sty}{% use microtype if available
  \usepackage[]{microtype}
  \UseMicrotypeSet[protrusion]{basicmath} % disable protrusion for tt fonts
}{}
\makeatletter
\@ifundefined{KOMAClassName}{% if non-KOMA class
  \IfFileExists{parskip.sty}{%
    \usepackage{parskip}
  }{% else
    \setlength{\parindent}{0pt}
    \setlength{\parskip}{6pt plus 2pt minus 1pt}}
}{% if KOMA class
  \KOMAoptions{parskip=half}}
\makeatother
\usepackage{xcolor}
\setlength{\emergencystretch}{3em} % prevent overfull lines
\setcounter{secnumdepth}{5}
% Make \paragraph and \subparagraph free-standing
\makeatletter
\ifx\paragraph\undefined\else
  \let\oldparagraph\paragraph
  \renewcommand{\paragraph}{
    \@ifstar
      \xxxParagraphStar
      \xxxParagraphNoStar
  }
  \newcommand{\xxxParagraphStar}[1]{\oldparagraph*{#1}\mbox{}}
  \newcommand{\xxxParagraphNoStar}[1]{\oldparagraph{#1}\mbox{}}
\fi
\ifx\subparagraph\undefined\else
  \let\oldsubparagraph\subparagraph
  \renewcommand{\subparagraph}{
    \@ifstar
      \xxxSubParagraphStar
      \xxxSubParagraphNoStar
  }
  \newcommand{\xxxSubParagraphStar}[1]{\oldsubparagraph*{#1}\mbox{}}
  \newcommand{\xxxSubParagraphNoStar}[1]{\oldsubparagraph{#1}\mbox{}}
\fi
\makeatother

\usepackage{color}
\usepackage{fancyvrb}
\newcommand{\VerbBar}{|}
\newcommand{\VERB}{\Verb[commandchars=\\\{\}]}
\DefineVerbatimEnvironment{Highlighting}{Verbatim}{commandchars=\\\{\}}
% Add ',fontsize=\small' for more characters per line
\usepackage{framed}
\definecolor{shadecolor}{RGB}{241,243,245}
\newenvironment{Shaded}{\begin{snugshade}}{\end{snugshade}}
\newcommand{\AlertTok}[1]{\textcolor[rgb]{0.68,0.00,0.00}{#1}}
\newcommand{\AnnotationTok}[1]{\textcolor[rgb]{0.37,0.37,0.37}{#1}}
\newcommand{\AttributeTok}[1]{\textcolor[rgb]{0.40,0.45,0.13}{#1}}
\newcommand{\BaseNTok}[1]{\textcolor[rgb]{0.68,0.00,0.00}{#1}}
\newcommand{\BuiltInTok}[1]{\textcolor[rgb]{0.00,0.23,0.31}{#1}}
\newcommand{\CharTok}[1]{\textcolor[rgb]{0.13,0.47,0.30}{#1}}
\newcommand{\CommentTok}[1]{\textcolor[rgb]{0.37,0.37,0.37}{#1}}
\newcommand{\CommentVarTok}[1]{\textcolor[rgb]{0.37,0.37,0.37}{\textit{#1}}}
\newcommand{\ConstantTok}[1]{\textcolor[rgb]{0.56,0.35,0.01}{#1}}
\newcommand{\ControlFlowTok}[1]{\textcolor[rgb]{0.00,0.23,0.31}{\textbf{#1}}}
\newcommand{\DataTypeTok}[1]{\textcolor[rgb]{0.68,0.00,0.00}{#1}}
\newcommand{\DecValTok}[1]{\textcolor[rgb]{0.68,0.00,0.00}{#1}}
\newcommand{\DocumentationTok}[1]{\textcolor[rgb]{0.37,0.37,0.37}{\textit{#1}}}
\newcommand{\ErrorTok}[1]{\textcolor[rgb]{0.68,0.00,0.00}{#1}}
\newcommand{\ExtensionTok}[1]{\textcolor[rgb]{0.00,0.23,0.31}{#1}}
\newcommand{\FloatTok}[1]{\textcolor[rgb]{0.68,0.00,0.00}{#1}}
\newcommand{\FunctionTok}[1]{\textcolor[rgb]{0.28,0.35,0.67}{#1}}
\newcommand{\ImportTok}[1]{\textcolor[rgb]{0.00,0.46,0.62}{#1}}
\newcommand{\InformationTok}[1]{\textcolor[rgb]{0.37,0.37,0.37}{#1}}
\newcommand{\KeywordTok}[1]{\textcolor[rgb]{0.00,0.23,0.31}{\textbf{#1}}}
\newcommand{\NormalTok}[1]{\textcolor[rgb]{0.00,0.23,0.31}{#1}}
\newcommand{\OperatorTok}[1]{\textcolor[rgb]{0.37,0.37,0.37}{#1}}
\newcommand{\OtherTok}[1]{\textcolor[rgb]{0.00,0.23,0.31}{#1}}
\newcommand{\PreprocessorTok}[1]{\textcolor[rgb]{0.68,0.00,0.00}{#1}}
\newcommand{\RegionMarkerTok}[1]{\textcolor[rgb]{0.00,0.23,0.31}{#1}}
\newcommand{\SpecialCharTok}[1]{\textcolor[rgb]{0.37,0.37,0.37}{#1}}
\newcommand{\SpecialStringTok}[1]{\textcolor[rgb]{0.13,0.47,0.30}{#1}}
\newcommand{\StringTok}[1]{\textcolor[rgb]{0.13,0.47,0.30}{#1}}
\newcommand{\VariableTok}[1]{\textcolor[rgb]{0.07,0.07,0.07}{#1}}
\newcommand{\VerbatimStringTok}[1]{\textcolor[rgb]{0.13,0.47,0.30}{#1}}
\newcommand{\WarningTok}[1]{\textcolor[rgb]{0.37,0.37,0.37}{\textit{#1}}}

\providecommand{\tightlist}{%
  \setlength{\itemsep}{0pt}\setlength{\parskip}{0pt}}\usepackage{longtable,booktabs,array}
\usepackage{calc} % for calculating minipage widths
% Correct order of tables after \paragraph or \subparagraph
\usepackage{etoolbox}
\makeatletter
\patchcmd\longtable{\par}{\if@noskipsec\mbox{}\fi\par}{}{}
\makeatother
% Allow footnotes in longtable head/foot
\IfFileExists{footnotehyper.sty}{\usepackage{footnotehyper}}{\usepackage{footnote}}
\makesavenoteenv{longtable}
\usepackage{graphicx}
\makeatletter
\newsavebox\pandoc@box
\newcommand*\pandocbounded[1]{% scales image to fit in text height/width
  \sbox\pandoc@box{#1}%
  \Gscale@div\@tempa{\textheight}{\dimexpr\ht\pandoc@box+\dp\pandoc@box\relax}%
  \Gscale@div\@tempb{\linewidth}{\wd\pandoc@box}%
  \ifdim\@tempb\p@<\@tempa\p@\let\@tempa\@tempb\fi% select the smaller of both
  \ifdim\@tempa\p@<\p@\scalebox{\@tempa}{\usebox\pandoc@box}%
  \else\usebox{\pandoc@box}%
  \fi%
}
% Set default figure placement to htbp
\def\fps@figure{htbp}
\makeatother

\usepackage{fvextra}
\DefineVerbatimEnvironment{Highlighting}{Verbatim}{breaklines,commandchars=\\\{\}}
\areaset[0.50in]{4.5in}{8in}
\makeatletter
\@ifpackageloaded{bookmark}{}{\usepackage{bookmark}}
\makeatother
\makeatletter
\@ifpackageloaded{caption}{}{\usepackage{caption}}
\AtBeginDocument{%
\ifdefined\contentsname
  \renewcommand*\contentsname{Table of contents}
\else
  \newcommand\contentsname{Table of contents}
\fi
\ifdefined\listfigurename
  \renewcommand*\listfigurename{List of Figures}
\else
  \newcommand\listfigurename{List of Figures}
\fi
\ifdefined\listtablename
  \renewcommand*\listtablename{List of Tables}
\else
  \newcommand\listtablename{List of Tables}
\fi
\ifdefined\figurename
  \renewcommand*\figurename{Figure}
\else
  \newcommand\figurename{Figure}
\fi
\ifdefined\tablename
  \renewcommand*\tablename{Table}
\else
  \newcommand\tablename{Table}
\fi
}
\@ifpackageloaded{float}{}{\usepackage{float}}
\floatstyle{ruled}
\@ifundefined{c@chapter}{\newfloat{codelisting}{h}{lop}}{\newfloat{codelisting}{h}{lop}[chapter]}
\floatname{codelisting}{Listing}
\newcommand*\listoflistings{\listof{codelisting}{List of Listings}}
\makeatother
\makeatletter
\makeatother
\makeatletter
\@ifpackageloaded{caption}{}{\usepackage{caption}}
\@ifpackageloaded{subcaption}{}{\usepackage{subcaption}}
\makeatother

\ifLuaTeX
\usepackage[bidi=basic]{babel}
\else
\usepackage[bidi=default]{babel}
\fi
\babelprovide[main,import]{english}
% get rid of language-specific shorthands (see #6817):
\let\LanguageShortHands\languageshorthands
\def\languageshorthands#1{}
\ifLuaTeX
  \usepackage[english]{selnolig} % disable illegal ligatures
\fi
\usepackage{bookmark}

\IfFileExists{xurl.sty}{\usepackage{xurl}}{} % add URL line breaks if available
\urlstyle{same} % disable monospaced font for URLs
\hypersetup{
  pdftitle={Book Title},
  pdfauthor={KING},
  pdflang={en},
  hidelinks,
  pdfcreator={LaTeX via pandoc}}


\title{Book Title}
\author{KING}
\date{2024-12-10}

\begin{document}
\frontmatter
\maketitle

\RecustomVerbatimEnvironment{verbatim}{Verbatim}{
   showspaces = false,
   showtabs = false,
   breaksymbolleft={},
   breaklines
}

\renewcommand*\contentsname{Table of contents}
{
\setcounter{tocdepth}{2}
\tableofcontents
}

\mainmatter
\bookmarksetup{startatroot}

\chapter*{KINGsNotese}\label{kingsnotese}
\addcontentsline{toc}{chapter}{KINGsNotese}

\markboth{KINGsNotese}{KINGsNotese}

\bookmarksetup{startatroot}

\chapter*{Intro to template}\label{intro-to-template}
\addcontentsline{toc}{chapter}{Intro to template}

\markboth{Intro to template}{Intro to template}

This document serves as a guide for a project designed to demonstrate
the process of generating PDF documents from Markdown using Quarto and
LaTeX. Additionally, it incorporates the use of GitHub Actions to
automate the generation process and GitHub Releases for storing the
final output. This repository has been structured as a GitHub Template,
allowing it to be easily used as a starting point for any book project
or documentation endeavor.

\part{Linux}

\chapter*{Commands}\label{commands}
\addcontentsline{toc}{chapter}{Commands}

\markboth{Commands}{Commands}

This document serves as a guide for a project designed to demonstrate
the process of generating PDF documents from Markdown using Quarto and
LaTeX. Additionally, it incorporates the use of GitHub Actions to
automate the generation process and GitHub Releases for storing the
final output. This repository has been structured as a GitHub Template,
allowing it to be easily used as a starting point for any book project
or documentation endeavor.

\part{Info}

\chapter*{apropos}\label{apropos}
\addcontentsline{toc}{chapter}{apropos}

\markboth{apropos}{apropos}

\section*{Overview}\label{overview}
\addcontentsline{toc}{section}{Overview}

\markright{Overview}

The \texttt{apropos} command searches the manual page names and
descriptions for a specific keyword. It's useful for finding commands
when you don't remember their exact names.

\section*{Syntax}\label{syntax}
\addcontentsline{toc}{section}{Syntax}

\markright{Syntax}

\begin{Shaded}
\begin{Highlighting}[]
\FunctionTok{apropos} \PreprocessorTok{[}\SpecialStringTok{options}\PreprocessorTok{]}\NormalTok{ keyword}
\end{Highlighting}
\end{Shaded}

\section*{Common Options}\label{common-options}
\addcontentsline{toc}{section}{Common Options}

\markright{Common Options}

\begin{itemize}
\tightlist
\item
  \texttt{-a}: Display only matches that satisfy all keywords
\item
  \texttt{-r}: Use regular expressions for searching
\item
  \texttt{-s\ sections}: Look only in given manual sections
\item
  \texttt{-l}: List only page names
\end{itemize}

\section*{Examples}\label{examples}
\addcontentsline{toc}{section}{Examples}

\markright{Examples}

\begin{Shaded}
\begin{Highlighting}[]
\CommentTok{\# Find commands related to passwords}
\FunctionTok{apropos}\NormalTok{ password}
\CommentTok{\# Shows all commands with "password" in their description}

\CommentTok{\# Search with multiple keywords}
\FunctionTok{apropos} \AttributeTok{{-}a}\NormalTok{ user password}
\CommentTok{\# Shows commands containing both "user" and "password"}

\CommentTok{\# Use regex pattern}
\FunctionTok{apropos} \AttributeTok{{-}r} \StringTok{\textquotesingle{}\^{}find.*\textquotesingle{}}
\CommentTok{\# Lists all commands starting with "find"}
\end{Highlighting}
\end{Shaded}

\section*{Tips}\label{tips}
\addcontentsline{toc}{section}{Tips}

\markright{Tips}

\begin{enumerate}
\def\labelenumi{\arabic{enumi}.}
\tightlist
\item
  Use when you can't remember the exact command name
\item
  Combine with \texttt{man} to read full documentation
\item
  More detailed than \texttt{whatis}
\item
  Great for discovering new commands
\end{enumerate}

\part{DevOps Introduction}

This document serves as a guide for a project designed to demonstrate
the process of generating PDF documents from Markdown using Quarto and
LaTeX. Additionally, it incorporates the use of GitHub Actions to
automate the generation process and GitHub Releases for storing the
final output. This repository has been structured as a GitHub Template,
allowing it to be easily used as a starting point for any book project
or documentation endeavor.

\chapter*{Containers}\label{containers}
\addcontentsline{toc}{chapter}{Containers}

\markboth{Containers}{Containers}

This document serves as a guide for a project designed to demonstrate
the process of generating PDF documents from Markdown using Quarto and
LaTeX. Additionally, it incorporates the use of GitHub Actions to
automate the generation process and GitHub Releases for storing the
final output. This repository has been structured as a GitHub Template,
allowing it to be easily used as a starting point for any book project
or documentation endeavor.


\backmatter


\end{document}
