% Options for packages loaded elsewhere
\PassOptionsToPackage{unicode}{hyperref}
\PassOptionsToPackage{hyphens}{url}
%
\documentclass[
  letterpaper,
  paper=6in:9in,
  pagesize=pdftex,
  headinclude=on,
  footinclude=on,
  12pt]{scrbook}

\usepackage{amsmath,amssymb}
\usepackage{iftex}
\ifPDFTeX
  \usepackage[T1]{fontenc}
  \usepackage[utf8]{inputenc}
  \usepackage{textcomp} % provide euro and other symbols
\else % if luatex or xetex
  \usepackage{unicode-math}
  \defaultfontfeatures{Scale=MatchLowercase}
  \defaultfontfeatures[\rmfamily]{Ligatures=TeX,Scale=1}
\fi
\usepackage{lmodern}
\ifPDFTeX\else  
    % xetex/luatex font selection
\fi
% Use upquote if available, for straight quotes in verbatim environments
\IfFileExists{upquote.sty}{\usepackage{upquote}}{}
\IfFileExists{microtype.sty}{% use microtype if available
  \usepackage[]{microtype}
  \UseMicrotypeSet[protrusion]{basicmath} % disable protrusion for tt fonts
}{}
\makeatletter
\@ifundefined{KOMAClassName}{% if non-KOMA class
  \IfFileExists{parskip.sty}{%
    \usepackage{parskip}
  }{% else
    \setlength{\parindent}{0pt}
    \setlength{\parskip}{6pt plus 2pt minus 1pt}}
}{% if KOMA class
  \KOMAoptions{parskip=half}}
\makeatother
\usepackage{xcolor}
\setlength{\emergencystretch}{3em} % prevent overfull lines
\setcounter{secnumdepth}{5}
% Make \paragraph and \subparagraph free-standing
\makeatletter
\ifx\paragraph\undefined\else
  \let\oldparagraph\paragraph
  \renewcommand{\paragraph}{
    \@ifstar
      \xxxParagraphStar
      \xxxParagraphNoStar
  }
  \newcommand{\xxxParagraphStar}[1]{\oldparagraph*{#1}\mbox{}}
  \newcommand{\xxxParagraphNoStar}[1]{\oldparagraph{#1}\mbox{}}
\fi
\ifx\subparagraph\undefined\else
  \let\oldsubparagraph\subparagraph
  \renewcommand{\subparagraph}{
    \@ifstar
      \xxxSubParagraphStar
      \xxxSubParagraphNoStar
  }
  \newcommand{\xxxSubParagraphStar}[1]{\oldsubparagraph*{#1}\mbox{}}
  \newcommand{\xxxSubParagraphNoStar}[1]{\oldsubparagraph{#1}\mbox{}}
\fi
\makeatother


\providecommand{\tightlist}{%
  \setlength{\itemsep}{0pt}\setlength{\parskip}{0pt}}\usepackage{longtable,booktabs,array}
\usepackage{calc} % for calculating minipage widths
% Correct order of tables after \paragraph or \subparagraph
\usepackage{etoolbox}
\makeatletter
\patchcmd\longtable{\par}{\if@noskipsec\mbox{}\fi\par}{}{}
\makeatother
% Allow footnotes in longtable head/foot
\IfFileExists{footnotehyper.sty}{\usepackage{footnotehyper}}{\usepackage{footnote}}
\makesavenoteenv{longtable}
\usepackage{graphicx}
\makeatletter
\newsavebox\pandoc@box
\newcommand*\pandocbounded[1]{% scales image to fit in text height/width
  \sbox\pandoc@box{#1}%
  \Gscale@div\@tempa{\textheight}{\dimexpr\ht\pandoc@box+\dp\pandoc@box\relax}%
  \Gscale@div\@tempb{\linewidth}{\wd\pandoc@box}%
  \ifdim\@tempb\p@<\@tempa\p@\let\@tempa\@tempb\fi% select the smaller of both
  \ifdim\@tempa\p@<\p@\scalebox{\@tempa}{\usebox\pandoc@box}%
  \else\usebox{\pandoc@box}%
  \fi%
}
% Set default figure placement to htbp
\def\fps@figure{htbp}
\makeatother

\usepackage{fvextra}
\DefineVerbatimEnvironment{Highlighting}{Verbatim}{breaklines,commandchars=\\\{\}}
\areaset[0.50in]{4.5in}{8in}
\makeatletter
\@ifpackageloaded{bookmark}{}{\usepackage{bookmark}}
\makeatother
\makeatletter
\@ifpackageloaded{caption}{}{\usepackage{caption}}
\AtBeginDocument{%
\ifdefined\contentsname
  \renewcommand*\contentsname{Inhaltsverzeichnis}
\else
  \newcommand\contentsname{Inhaltsverzeichnis}
\fi
\ifdefined\listfigurename
  \renewcommand*\listfigurename{Abbildungsverzeichnis}
\else
  \newcommand\listfigurename{Abbildungsverzeichnis}
\fi
\ifdefined\listtablename
  \renewcommand*\listtablename{Tabellenverzeichnis}
\else
  \newcommand\listtablename{Tabellenverzeichnis}
\fi
\ifdefined\figurename
  \renewcommand*\figurename{Abbildung}
\else
  \newcommand\figurename{Abbildung}
\fi
\ifdefined\tablename
  \renewcommand*\tablename{Tabelle}
\else
  \newcommand\tablename{Tabelle}
\fi
}
\@ifpackageloaded{float}{}{\usepackage{float}}
\floatstyle{ruled}
\@ifundefined{c@chapter}{\newfloat{codelisting}{h}{lop}}{\newfloat{codelisting}{h}{lop}[chapter]}
\floatname{codelisting}{Listing}
\newcommand*\listoflistings{\listof{codelisting}{Listingverzeichnis}}
\makeatother
\makeatletter
\makeatother
\makeatletter
\@ifpackageloaded{caption}{}{\usepackage{caption}}
\@ifpackageloaded{subcaption}{}{\usepackage{subcaption}}
\makeatother

\ifLuaTeX
\usepackage[bidi=basic]{babel}
\else
\usepackage[bidi=default]{babel}
\fi
\babelprovide[main,import]{ngerman}
% get rid of language-specific shorthands (see #6817):
\let\LanguageShortHands\languageshorthands
\def\languageshorthands#1{}
\ifLuaTeX
  \usepackage[german]{selnolig} % disable illegal ligatures
\fi
\usepackage{bookmark}

\IfFileExists{xurl.sty}{\usepackage{xurl}}{} % add URL line breaks if available
\urlstyle{same} % disable monospaced font for URLs
\hypersetup{
  pdftitle={Book Title},
  pdfauthor={Your Name},
  pdflang={de},
  hidelinks,
  pdfcreator={LaTeX via pandoc}}


\title{Book Title}
\author{Your Name}
\date{2024-12-04}

\begin{document}
\frontmatter
\maketitle

\RecustomVerbatimEnvironment{verbatim}{Verbatim}{
   showspaces = false,
   showtabs = false,
   breaksymbolleft={},
   breaklines
}

\renewcommand*\contentsname{Inhaltsverzeichnis}
{
\setcounter{tocdepth}{2}
\tableofcontents
}

\mainmatter
\bookmarksetup{startatroot}

\chapter*{Generating PDF Documents with Quarto, LaTeX, and GitHub
Actions}\label{generating-pdf-documents-with-quarto-latex-and-github-actions}
\addcontentsline{toc}{chapter}{Generating PDF Documents with Quarto,
LaTeX, and GitHub Actions}

\markboth{Generating PDF Documents with Quarto, LaTeX, and GitHub
Actions}{Generating PDF Documents with Quarto, LaTeX, and GitHub
Actions}

\bookmarksetup{startatroot}

\chapter{Introduction}\label{introduction}

This document serves as a guide for a project designed to demonstrate
the process of generating PDF documents from Markdown using Quarto and
LaTeX. Additionally, it incorporates the use of GitHub Actions to
automate the generation process and GitHub Releases for storing the
final output. This repository has been structured as a GitHub Template,
allowing it to be easily used as a starting point for any book project
or documentation endeavor.

\bookmarksetup{startatroot}

\chapter{Overview of Technologies}\label{overview-of-technologies}

Before diving into the process, let's briefly overview the key
technologies used in this project:

\subsection{Quarto}\label{quarto}

Quarto is an open-source scientific and technical publishing system
built on Pandoc. It allows users to convert documents written in
Markdown, R Markdown, Jupyter, or Qmd to various formats including HTML,
PDF, and EPUB.

\subsection{LaTeX}\label{latex}

LaTeX is a high-quality typesetting system; it includes features
designed for the production of technical and scientific documentation.
LaTeX is the de facto standard for the communication and publication of
scientific documents.

\subsection{GitHub Actions}\label{github-actions}

GitHub Actions enables automation of all your software workflows, now
with world-class CI/CD (Continuous Integration/Continuous Deployment).
Build, test, and deploy your code right from GitHub.

\subsection{GitHub Release}\label{github-release}

GitHub Releases are a way to ship software to your users. It's a GitHub
feature that makes it easy to bundle source code, release notes, and
links to binary files for others to use.

\bookmarksetup{startatroot}

\chapter{Project Structure}\label{project-structure}

The repository is structured as follows:

\begin{itemize}
\tightlist
\item
  \texttt{content/}: This directory contains the Markdown files (.qmd or
  .md) that make up the content of your book or document.
\item
  \texttt{.github/workflows/}: This directory contains the GitHub
  Actions workflow files that automate the build process.
\item
  \texttt{\_quarto.yml}: The Quarto configuration file that specifies
  how the Markdown should be converted to LaTeX and then to PDF. It also
  can generate .docx and .epub files.
\end{itemize}

\bookmarksetup{startatroot}

\chapter{Workflow}\label{workflow}

The process of generating a PDF document from Markdown using this
project/template involves the following steps:

\begin{enumerate}
\def\labelenumi{\arabic{enumi}.}
\item
  \textbf{Writing Content in Markdown}: Authors can write the content of
  their book or document in Markdown, storing these files in the
  \texttt{content/} directory.
\item
  \textbf{Configuring Quarto}: In the \texttt{\_quarto.yml} file,
  authors specify the settings for Quarto to convert Markdown to LaTeX.
  This includes configurations such as document title, author, date, and
  LaTeX-specific settings.
\item
  \textbf{Commit, Tag and Push}: Add the changes \texttt{git\ add\ .},
  commit them \texttt{git\ commit\ -m\ "your\ commit\ comment"}, create
  a git tag \texttt{git\ tag\ -a\ v1.0.0\ -m\ "Release\ version\ 1.0.0"}
  and push it \texttt{git\ push\ origin\ v1.0.0}.
\item
  \textbf{Using GitHub Actions for Automation}: When changes are pushed
  with a git-tag to the repository, GitHub Actions, as defined in
  \texttt{.github/workflows/}, automatically triggers the build process.
  The workflow installs Quarto, checks out the repository content, and
  runs the Quarto command to convert Markdown to LaTeX and then to PDF.
\item
  \textbf{Storing the Result with GitHub Release}: Once the PDF is
  generated, the GitHub Actions workflow automatically creates a new
  release and attaches the PDF document to it. This makes the document
  accessible to users and provides version control for each release.
\end{enumerate}

\bookmarksetup{startatroot}

\chapter{Using the Repository as a GitHub
Template}\label{using-the-repository-as-a-github-template}

To use this repository as a starting point for your book project, follow
these steps:

\begin{enumerate}
\def\labelenumi{\arabic{enumi}.}
\tightlist
\item
  Click on the `Use this template' button on the repository page.
\item
  Choose a name and description for your new repository and select
  whether it should be public or private.
\item
  Click `Create repository from template' to create a new repository in
  your account with the content and structure of this project.
\end{enumerate}

\bookmarksetup{startatroot}

\chapter{Conclusion}\label{conclusion}

This project provides a streamlined and automated approach to converting
Markdown documents into professional-looking PDFs using Quarto, LaTeX,
and GitHub Actions. By leveraging GitHub Releases, it also offers a
convenient way to distribute and version-control the generated
documents. As a GitHub Template, it is readily available for anyone
looking to start a book project or any comprehensive documentation.


\backmatter


\end{document}
